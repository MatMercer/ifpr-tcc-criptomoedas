%% abtex2-modelo-projeto-pesquisa.tex, v-1.9.6 laurocesar
%% Copyright 2012-2016 by abnTeX2 group at http://www.abntex.net.br/ 
%%
%% This work may be distributed and/or modified under the
%% conditions of the LaTeX Project Public License, either version 1.3
%% of this license or (at your option) any later version.
%% The latest version of this license is in
%%   http://www.latex-project.org/lppl.txt
%% and version 1.3 or later is part of all distributions of LaTeX
%% version 2005/12/01 or later.
%%
%% This work has the LPPL maintenance status `maintained'.
%% 
%% The Current Maintainer of this work is the abnTeX2 team, led
%% by Lauro César Araujo. Further information are available on 
%% http://www.abntex.net.br/
%%
%% This work consists of the files abntex2-modelo-projeto-pesquisa.tex
%% and abntex2-modelo-references.bib
%%

% ------------------------------------------------------------------------
% ------------------------------------------------------------------------
% abnTeX2: Modelo de Projeto de pesquisa em conformidade com 
% ABNT NBR 15287:2011 Informação e documentação - Projeto de pesquisa -
% Apresentação 
% ------------------------------------------------------------------------ 
% ------------------------------------------------------------------------

\documentclass[
	% -- opções da classe memoir --
	article,			% indica que é um artigo acadêmico
	12pt,				% tamanho da fonte
	openright,			% capítulos começam em pág ímpar (insere página vazia caso preciso)
	oneside,			% para impressão em recto. Oposto a twoside
	%twoside,			% para impressão em recto e verso. Oposto a oneside
	a4paper,			% tamanho do papel. 
	% -- opções da classe abntex2 --
	chapter=TITLE,		% títulos de capítulos convertidos em letras maiúsculas
	section=TITLE,		% títulos de seções convertidos em letras maiúsculas
	subsection=TITLE,	% títulos de subseções convertidos em letras maiúsculas
	subsubsection=TITLE,% títulos de subsubseções convertidos em letras maiúsculas
	subsubsubsection=TITLE, % títulos de subsubsubseções em letras maiúsculas
	% -- opções do pacote babel --
	english,			% idioma adicional para hifenização
	%french,			% idioma adicional para hifenização
	%spanish,			% idioma adicional para hifenização
	brazil,				% o último idioma é o principal do documento
	]{abntex2}

% ---
% PACOTES
% ---

% ---
% Pacotes fundamentais 
% ---
\usepackage{lmodern}			% Usa a fonte Latin Modern
\usepackage[T1]{fontenc}		% Selecao de codigos de fonte.
\usepackage[utf8]{inputenc}		% Codificacao do documento (conversão automática dos acentos)
\usepackage{indentfirst}		% Indenta o primeiro parágrafo de cada seção.
\usepackage{color}				% Controle das cores
\usepackage{graphicx}			% Inclusão de gráficos
\usepackage{microtype} 			% para melhorias de justificação
%\usepackage[none]{hyphenat} 			% Desativa ifenização
% ---

% ---
% Pacotes adicionais, usados apenas no âmbito do Modelo Canônico do abnteX2
% ---
\usepackage{lipsum}				% para geração de dummy text
% ---

% ---
% Pacotes de citações
% ---
\usepackage[brazilian,hyperpageref]{backref}	 % Paginas com as citações na bibl
\usepackage[alf]{abntex2cite}	% Citações padrão ABNT

% --- 
% CONFIGURAÇÕES DE PACOTES
% --- 

% ---
% Configurações do pacote backref
% Usado sem a opção hyperpageref de backref
\renewcommand{\backrefpagesname}{Citado na(s) página(s):~}
% Texto padrão antes do número das páginas
\renewcommand{\backref}{}
% Define os textos da citação
\renewcommand*{\backrefalt}[4]{%
	\ifcase #1 %
		Nenhuma citação no texto.%
	\or
		Citado na página #2.%
	\else
		Citado #1 vezes nas páginas #2.%
	\fi}%
% ---

% ---
% Configurações do modelo IFPR
\usepackage{abntex2ifpr}

% ---
% Informações de dados para CAPA, FOLHA DE ROSTO e FOLHA DE APROVAÇÃO
% ---
\titulo{Modelo para trabalhos acadêmicos no IFPR com \abnTeX}
\autor{Mateus Mercer e Equipe \abnTeX}
\local{Londrina}
\data{2018}
\orientador{INSIRA O ORIENTADOR}
%\coorientador{INSIRA O COORIENTADOR}
\convidadoum{INSIRA O CONVIDADO 1}
\convidadodois{INSIRA O CONVIDADO 2}
\curso{Curso Superior de Tecnologia em Análise e Desenvolvimento de Sistemas}
% O preambulo deve conter o tipo do trabalho, o objetivo, 
% o nome da instituição e a área de concentração 
\preambulo{Trabalho de Conclusão de Curso apresentado ao \imprimircurso do Instituto 
Federal do Paraná \@\textendash\@ Campus Londrina, como 
requisito parcial de avaliação.}
% ---

% ---
% Configurações de aparência do PDF final

% alterando o aspecto da cor azul
\definecolor{blue}{RGB}{41,5,195}

% informações do PDF
\makeatletter
\hypersetup{%
     	%pagebackref=true,
		pdftitle={\@title}, 
		pdfauthor={\@author},
    	pdfsubject={\imprimirpreambulo},
	    pdfcreator={LaTeX with abnTeX2},
		pdfkeywords={abnt}{latex}{abntex}{abntex2}{projeto de pesquisa}, 
		colorlinks=false,       		% false: boxed links; true: colored links
    	linkcolor=blue,          	% color of internal links
    	citecolor=blue,        		% color of links to bibliography
    	filecolor=magenta,      		% color of file links
		urlcolor=blue,
		bookmarksdepth=4
}
\makeatother
% --- 

% --- 
% Espaçamentos entre linhas e parágrafos 
% --- 

% O tamanho do parágrafo é dado por:
\setlength{\parindent}{1.3cm}

% Controle do espaçamento entre um parágrafo e outro:
\setlength{\parskip}{0.2cm}  % tente também \onelineskip

% ---
% compila o indice
% ---
\makeindex
% ---

% ---
% Aonde as imagens estao
% ---
\graphicspath{{imagens/}}
% ---

% ----
% Início do documento
% ----
\begin{document}

% Seleciona o idioma do documento (conforme pacotes do babel)
%\selectlanguage{english}
\selectlanguage{brazil}

% Retira espaço extra obsoleto entre as frases.
\frenchspacing 

% ----------------------------------------------------------
% ELEMENTOS PRÉ-TEXTUAIS
% ----------------------------------------------------------
% \pretextual

% ---
% Capa
% ---
\imprimircapa
% ---

% ---
% Folha de rosto
% ---
\imprimirfolhaderosto
% ---

% ---
% Inserir folha de aprovação
% ---

% Isto é um exemplo de Folha de aprovação, elemento obrigatório da NBR
% 14724/2011 (seção 4.2.1.3). Você pode utilizar este modelo até a aprovação
% do trabalho. Após isso, substitua todo o conteúdo deste arquivo por uma
% imagem da página assinada pela banca com o comando abaixo:
%
% \begin{folhadeaprovacao}
% \includepdf{folhadeaprovacao_final.pdf}
% \end{folhadeaprovacao}
%
\begin{folhadeaprovacao}
  \begin{center}
    \begin{center}
      \ABNTEXchapterfont\bfseries FOLHA DE APROVAÇÃO
      \par
        \vspace*{1.5cm}
        {\normalfont\ABNTEXchapterfontsize\MakeUppercase\imprimirautor}
        \vspace*{1.5cm}
        \par
        {\normalfont\ABNTEXchapterfontsize\MakeUppercase\imprimirtitulo}
    \end{center}
    \vspace*{\fill}
    
    \hspace{.45\textwidth}
    \begin{minipage}{.5\textwidth}
        \imprimirpreambulo
    \end{minipage}%
    \vspace*{\fill}
   \end{center}

   \assinatura{\textbf{\imprimirorientador} \\ Orientador} 
   \assinatura{\textbf{\imprimirconvidadoum} \\ Convidado 1}
   \assinatura{\textbf{\imprimirconvidadodois} \\ Convidado 2}
   %\assinatura{\textbf{Professor} \\ Convidado 3}
   %\assinatura{\textbf{Professor} \\ Convidado 4}
      
   \vspace{3cm}
  \begin{center}
   \imprimirlocal, 22 de Janeiro de 1999.
  \end{center}
  
  \pagebreak
\end{folhadeaprovacao}
% ---

% ---
% NOTA DA ABNT NBR 15287:2011, p. 4:
%  ``Se exigido pela entidade, apresentar os dados curriculares do autor em
%     folha ou página distinta após a folha de rosto.''
% ---

% ---
% inserir lista de ilustrações
% ---
\pdfbookmark[0]{\listfigurename}{lof}
\listoffigures*
\cleardoublepage
% ---

% ---
% inserir lista de tabelas
% ---
\pdfbookmark[0]{\listtablename}{lot}
\listoftables*
\cleardoublepage
% ---

% ---
% inserir lista de abreviaturas e siglas
% ---
\begin{siglas}
  \item[ABNT] Associação Brasileira de Normas Técnicas
  \item[abnTeX] ABsurdas Normas para TeX
\end{siglas}
% ---

% ---
% inserir lista de símbolos
% ---
\begin{simbolos}
  \item[$ \Gamma $] Letra grega Gama
  \item[$ \Lambda $] Lambda
  \item[$ \zeta $] Letra grega minúscula zeta
  \item[$ \in $] Pertence
\end{simbolos}
% ---

% ---
% inserir o sumario
% ---
\pdfbookmark[0]{\contentsname}{toc}
\tableofcontents*
\cleardoublepage
% ---


% ----------------------------------------------------------
% ELEMENTOS TEXTUAIS
% ----------------------------------------------------------
\textual

% Introdução
\section{Introdução}

\section{Contextualização e Análise Competitiva}

\subsection{Criptomoedas}

Sua decentralização e independência de uma instituição financeira
\cite{Nakamoto2008} tornou as taxas de transferências em transações
cambiais menores, além de aumentar a velocidade com que essas
transações eram efetivadas. O investimento interessou grande parte dos
participantes deste novo mercado \cite{Prado2017}.

\subsubsection{Definição}

Uma Criptomoeda é um sistema \emph{peer-to-peer} onde o controle da
moeda é feito por propriedades matemáticas e não por uma instituição
centralizada. Cada usuário é identificado por uma representação de
chave pública, sua carteira, e faz uma transação com sua chave
privada. Em vez de existir um balanço em uma conta, existe uma lista
de transações feitas desde o início do sistema, chamada de
\emph{Blockchain} \cite{Weber2012}.

O \emph{Blockchain}, também referenciado como ``protocolo de
confiança'', usa a decentralização de uma instituição financeira como
uma medida de segurança. Cada transação deve ser validada
criptograficamente por todos os \emph{nodes} de uma rede antes de
entrar para o \empH{Blockchain}. Diferente de um banco, não existe uma
instituição que valida as transações, a própria rede faz esta
verificação \cite{LChicarino}.


% ----------------------------------------------------------
% Referências bibliográficas (nome do arquivo de referencias, sem o ".bib"
% ----------------------------------------------------------
\bibliography{referencias}

\end{document}
